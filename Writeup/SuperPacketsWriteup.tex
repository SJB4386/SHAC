\documentclass[11pt]{article}
\usepackage{setspace}
\usepackage[letterpaper,top=1in,bottom=1in,left=1in,right=1in]{geometry}
\usepackage[parfill]{parskip}
\usepackage{listings}
\usepackage{arydshln}


\title{Super-Packets High-Availability Cluster Project}
\author{Spencer Berg, William Johnson, Jonathan Rogers}
\date{7 March 2018}

\begin{document}
\maketitle

\thispagestyle{empty}

\begin{doublespace}
\section{How This Works}

\subsection{Header Structure}
\begin{center}
\textbf{SHAC 1.0}
\begin{tabular}{|c|c|c|}
\hline
Field & Length (bytes) & Description \\
\hline
VERSION & 6 & The version of SHAC the packet originated from \\
\hline
NODE\_COUNT & 1 & The number of nodes contained in the packet \\
\hline
 &  & Identifiers for the type of node that sent the packet.\\
FLAGS & 1 & The flags are CLIENT, SERVER, and PEER \\ 
 & & in this order where CLIENT is the LSB \\
\hline
NODES & $13*$NODE\_COUNT& \\
\hdashline
IP & 4 & The IPv4 address of the node \\
\hdashline
TIMESTAMP & 8 & UNIX timestamp which give the last time the sender\\
& & received a packet from the node \\
\hdashline
AVAILABILITY & 1 & The status of the node, defined by the LSB \\
\hline
\end{tabular}
\end{center}

\subsection{Client-Server}

The server works by first initializing a thread that listens for any connections, then initializes a second thread that will check the node list when new content is added to the server. A client begins, encodes a SHAC packet with its information and sends it to the server allowing a connection to be formed. The server, after receiving the SHAC packet decodes it, and checks to see if it is a new connection or has changed the servers master list in any way. The SHAC packet is then added to the server's node list and the thread that was initialized earlier keeps tabs on it to make sure it remains active every 30 seconds. The server then sends out a SHAC packet that contains the updated list of nodes. If another client had already connected to the server then it would recieve the packet, decode it, and check to see if there was a change to the client's list of nodes. If there was the client updates its local list and keeps sending out availability signals every random interval between 0 and 30 seconds. 

\subsection{P2P}

The P2P networks is somewhat similar to the above client server set up. Each node acts as a hybrid client-server to achieve a fully connected P2P network. When the first peer, peer1, is started it begins by initializing itself and begins broadcasting its list of known peers, only itself. Then when a second peer joins the network, peer2, it connects to peer1 by sending a packet. Peer1 receives the packet and much like Client-Server, decodes, analyzes, and updates its local list appropriately. The new packets broadcast by peer1 will now contain the ip address and availability of peer2. Now if a third peer, peer3, joins the network by connecting to peer2 the same process will occur. Then when peer2 broadcasts its new updated packet, which peer1 will receive, it will contain the ip address and availability of peer3. Peer1 will once again decode, analyze and update its local node and then establish a connection with peer3 in addition to its connection with peer2. In this way the network stays totally connected with every node being aware of every other node, each new node is known in two transmissions, and the availability of every node is known after all local timers have depleted meaning no network action is needed to inform the network of an unavailable node.

\section{The SHAC Protocol}

\subsection{Pros}
The Small High Availability Cluster (SHAC) Protocol has several positive traits. First off the design itself is light weight meaning it can be used on small devices that are running Java. This means that it is excellent for High Availability Clusters whose purpose might be on the IoT side of networking. Secondly while the design uses an entire byte for simplicity in the flags section, the reality is that those bits can easily be converted to be used for other purposes. Thus the SHAC protocol has the ability to grow as new use cases arise. Thirdly on the P2P side of things, each transmission sends all known packets to all other peers. That means any peer joining the network will be known by all other peers in two transmissions. The first being its initial connection to the network and the second being the peer it connected to informing the rest of the network.

\subsection{Cons} 
No protocol is a panacea and the SHAC Protocol is no different. First in P2P mode every node is connected and sends a list of its peers to every other node. In a small cluster this works well but as the cluster grows this increases the amount of traffic considerably. Working with a large cluster would require a different approach to writing a protocol as scaling would be a more significant issue. Secondly using a static time of 30 seconds has some drawbacks especially with UDP. Namely if a packet is dropped it takes half a minute before anything can be done. An improvement might be either reducing the window of time to 20 seconds, or using a static check in time rather than a random one. 

\section{Team Contributions}

The whole team met to discuss the design of the protocol, then split the task of writing code.
The protocol code was written by William.
The P2P and client classes were were written by Spencer.
The Server class was written by Jonathan.
The group collectively worked out any flaws found in the others' code.

\section{Code}
\end{doublespace}
\begin{tiny}

\begin{lstinputlisting}[language=Java]{../src/SHAC/protocol/NodeType.java}
\end{lstinputlisting}
\begin{lstinputlisting}[language=Java]{../src/SHAC/protocol/SHACData.java}
\end{lstinputlisting}
\begin{lstinputlisting}[language=Java]{../src/SHAC/protocol/SHACNode.java}
\end{lstinputlisting}
\begin{lstinputlisting}[language=Java]{../src/SHAC/protocol/SHACProtocol.java}
\end{lstinputlisting}
\begin{lstinputlisting}[language=Java]{../src/SHAC/protocol/SHACProtocolTest.java}
\end{lstinputlisting}
\begin{lstinputlisting}[language=Java]{../src/SHAC/client/SHACClient.java}
\end{lstinputlisting}
\begin{lstinputlisting}[language=Java]{../src/SHAC/server/SHACServer.java}
\end{lstinputlisting}
\begin{lstinputlisting}[language=Java]{../src/SHAC/peer/SHACPeer.java}
\end{lstinputlisting}

\end{tiny}

\end{document}
