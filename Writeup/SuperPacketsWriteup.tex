\documentclass[11pt]{article}
\usepackage{setspace}
\usepackage[letterpaper,top=1in,bottom=1in,left=1in,right=1in]{geometry}
\usepackage[parfill]{parskip}
\usepackage{listings}
\usepackage{arydshln}


\title{Super-Packets High-Availability Cluster Project}
\author{Spencer Berg, William Johnson, Jonathan Rogers}
\date{7 March 2018}

\begin{document}
\maketitle

\thispagestyle{empty}

\begin{doublespace}
\section{How This Works}

\subsection{Header Structure}
\begin{center}
\textbf{SHAC 1.0}
\begin{tabular}{|c|c|c|}
\hline
Field & Length (bytes) & Description \\
\hline
VERSION & 6 & The version of SHAC the packet originated from \\
\hline
NODE\_COUNT & 1 & The number of nodes contained in the packet \\
\hline
 &  & Identifiers for the type of node that sent the packet.\\
FLAGS & 1 & The flags are CLIENT, SERVER, and PEER \\ 
 & & in this order where CLIENT is the LSB \\
\hline
NODES & $13*$NODE\_COUNT& \\
\hdashline
IP & 4 & The IPv4 address of the node \\
\hdashline
TIMESTAMP & 8 & UNIX timestamp which give the last time the sender\\
& & received a packet from the node \\
\hdashline
AVAILABILITY & 1 & The status of the node, defined by the LSB \\
\hline
\end{tabular}
\end{center}

\subsection{Client-Server}



\subsection{P2P}



\section{Why This is Good Design}



\section{Team Contributions}

The whole team met to discuss the design of the protocol, then split the task of writing code.
The protocol code was written by William.
The P2P and client classes were were written by Spencer.
The Server class was written by Jonathan.
The group collectively worked out any flaws found in the others' code.

\section{Code}
\end{doublespace}
\begin{tiny}

\begin{lstinputlisting}[language=Java]{../src/SHAC/protocol/NodeType.java}
\end{lstinputlisting}
\begin{lstinputlisting}[language=Java]{../src/SHAC/protocol/SHACData.java}
\end{lstinputlisting}
\begin{lstinputlisting}[language=Java]{../src/SHAC/protocol/SHACNode.java}
\end{lstinputlisting}
\begin{lstinputlisting}[language=Java]{../src/SHAC/protocol/SHACProtocol.java}
\end{lstinputlisting}
\begin{lstinputlisting}[language=Java]{../src/SHAC/protocol/SHACProtocolTest.java}
\end{lstinputlisting}
\begin{lstinputlisting}[language=Java]{../src/SHAC/client/SHACClient.java}
\end{lstinputlisting}
\begin{lstinputlisting}[language=Java]{../src/SHAC/server/SHACServer.java}
\end{lstinputlisting}
\begin{lstinputlisting}[language=Java]{../src/SHAC/peer/SHACPeer.java}
\end{lstinputlisting}

\end{tiny}

\end{document}
