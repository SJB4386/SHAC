\documentclass[11pt]{article}
\usepackage{setspace}
\usepackage[letterpaper,top=1in,bottom=1in,left=1.25in,right=1.25in]{geometry}
\usepackage[parfill]{parskip}

\title{High-Availability Cluster Project}
\author{Super-Packets}
\date{7 March 2018}

\begin{document}
\maketitle

\thispagestyle{empty}

\begin{doublespace}
  \section{How This Works}

  Lorem ipsum dolor sit amet, consectetur adipiscing elit. Curabitur
  enim elit, fermentum ac porttitor eget, ullamcorper eu
  ipsum. Vivamus risus ex, gravida ut egestas sed, vestibulum sed
  metus. Maecenas a tellus mi. Nam sit amet nisl non augue tristique
  efficitur.

  Nam eget tincidunt odio. Nulla cursus arcu vel urna congue, a tempus
  elit ultrices. Donec nec ornare enim. Nullam ultrices libero dui,
  nec consequat tellus facilisis eget. Fusce nec massa a elit ornare
  venenatis. Nunc tellus ligula, luctus sed fringilla nec, venenatis
  non orci. Pellentesque bibendum turpis eleifend nisi placerat
  semper.

  \section{Why This is Good Design}

  Sed lacus justo, ``tempus in'', tincidunt vel, accumsan eget,
  pede. Nullam massa eros, tempus sed, ultricies quis, nonummy vitae,
  eros. Nulla dapibus dictum metus. Nulla
  vulputate vestibulum purus.

  \section{Team Contributions}
  
  Fusce dolor nisi, mollis quis, malesuada a, placerat sed,
  dolor. Vestibulum ut nisl eu elit vestibulum
  aliquet\cite. Sed placerat sollicitudin
  nisl. Aliquam erat volutpat. Sed elit arcu, bibendum
  id, dictum eget, volutpat vitae, felis. Vestibulum at augue. Nulla a
  velit elementum turpis pellentesque
  blandit\cite. Integer turpis. Ut ac nulla nec massa
  facilisis cursus. Mauris condimentum, est id semper varius, ante
  enim porta quam, id feugiat lacus dolor vel
  turpis\cite{beck2008}. Suspendisse vulputate viverra nisl. Praesent
  tincidunt metus id diam.

  \section{Code}

\begin{lstlisting}
package SHAC.client;

import java.io.IOException;
import java.net.*;
import java.util.ArrayList;
import java.util.Random;
import java.util.Timer;
import java.util.TimerTask;
import SHAC.protocol.*;

public class SHACClient extends Thread {
    private Timer timer;
    private Random rand;
    private DatagramSocket socket;
    public ArrayList<SHACNode> nodes;
    public String serverIP;

    public SHACClient() {
        serverIP = "localhost";
        initializeClient();
    }

    public SHACClient(String serverIP) {
        this.serverIP = serverIP;
        initializeClient();
    }

    private void initializeClient() {
        timer = new Timer();
        rand = new Random();
        nodes = new ArrayList<SHACNode>();
        try {
            socket = new DatagramSocket();
        } catch (SocketException e) {
            e.printStackTrace();
        }
    }

    private void runClient() {
        start();
        startSendingUpdates();
    }

    public void run() {
        listenForUpdates();
    }

    private void startSendingUpdates() {
        // Send an update, then set a timer to do it again
        sendUpdate();
        timer.schedule(new TimerTask() {
            public void run() {
                startSendingUpdates();
            }
        }, rand.nextInt(30) * 1000);
    }

    private void sendUpdate() {
        // Send a packet to greet server
        try {
            InetAddress IPAddress = InetAddress.getByName(serverIP);
            SHACData update = new SHACData(0, NodeType.CLIENT);
            byte[] data = SHACProtocol.encodePacketData(update);
            DatagramPacket sendPacket = new DatagramPacket(data, data.length, IPAddress, SHACProtocol.SHAC_SOCKET);
            socket.send(sendPacket);
            System.out.println("Message sent from client");
        } catch (UnknownHostException e) {
            e.printStackTrace();
        } catch (SocketException e) {
            e.printStackTrace();
        } catch (IOException e) {
            e.printStackTrace();
        }
    }

    public void listenForUpdates() {
        while (true) {
            try {
                byte[] incomingData = new byte[1024];
                DatagramPacket incomingPacket = new DatagramPacket(incomingData, incomingData.length);
                socket.receive(incomingPacket);
                SHACData data = SHACProtocol.decodePacketData(incomingPacket.getData());
                if (data.nodeTypeFlag == NodeType.SERVER) {
                    nodes = data.nodes;
                }
                System.out.println("Received availability update from server.");
            } catch (UnknownHostException e) {
                e.printStackTrace();
            } catch (SocketException e) {
                e.printStackTrace();
            } catch (IOException e) {
                e.printStackTrace();
            }
        }
    }

    public void printAvailableNodes() {
        // Return status of each node. Change return type to what's appropriate
        System.out.println("Available nodes:");
        for (SHACNode n : nodes) {
            System.out.println(n.toString());
        }
    }

    public static void main(String[] args) {
        SHACClient s;
        if (args.length == 0) {
            s = new SHACClient();
        } else {
            s = new SHACClient(args[0]);
        }
        s.runClient();
    }

}

\end{lstlisting}

\end{doublespace}
\end{document}
